%\usepackage{amsmath,amsthm,amsfonts,amscd,amssymb,amstext} 
\usepackage{amsmath,amsfonts,amscd,amssymb,amstext} 
\usepackage{mathrsfs}
				% Some packages to write mathematics.
\usepackage{overpic}
\usepackage{eucal} 	 	% Euler fonts
\usepackage{verbatim}      	% Allows quoting source with commands.
\usepackage{makeidx}       	% Package to make an index.
%\usepackage{citesort}         	% 
\usepackage{subfig}
\usepackage{graphicx}
\usepackage{url}		% Allows good typesetting of web URLs.
\usepackage{multirow}
%\usepackage[linesnumbered, noline]{algorithm2e}
%\usepackage[boxed, noline]{algorithm2e}
\usepackage[ruled]{algorithm2e}
\usepackage{booktabs}
\usepackage{tabularx}
\usepackage{array}
\usepackage{adjustbox}
\usepackage{color}
\usepackage{tikz}
% To balance the columns at the end
\usepackage{balance}
\usepackage[mediumspace,mediumqspace,Grey,squaren]{SIunits}
%\usepackage{draftcopy}		% Uncomment this line to have the
				% word, "DRAFT," as a background
				% "watermark" on all of the pages of
				% of your draft versions. When ready
				% to generate your final copy, re-comment
				% it out with a percent sign to remove
				% the word draft before you re-run
				% Makediss for the last time.
\usepackage[normalem]{ulem}

%\usepackage{cite}
\usepackage{balance}
\usepackage{color}

\definecolor{bostonuniversityred}{rgb}{0.8, 0.0, 0.0}

%------------------------------------------------------------
% Custom commands
%------------------------------------------------------------
%\newcommand{\old}[1]{\textcolor{blue}{\sout{#1}}}
\newcommand{\old}[1]{}

%\newcommand{\red}[1]{\textcolor{red}{#1}}
\newcommand{\red}[1]{\textcolor{bostonuniversityred}{#1}}
\definecolor{light-gray}{gray}{0.75}
\newcommand{\gray}[1]{\textcolor{light-gray}{#1}}
%\newcommand{\red}[1]{#1}

%\newcommand{\reconstruct}{RECONSTRUCT\textsuperscript{\texttrademark}}
%\newcommand{\tm}{\textsuperscript{\texttrademark}}
\newcommand{\tm}{$^1$}
\newcommand{\addref}{\red{REF}}
\newcommand{\mathtext}[1]{\text{#1}}
\newcommand{\etal}{et al.\ }
\newcommand{\algorithmspace}{\vspace{8pt}}
\newcommand{\glfs}{\emph{glfs}}

\renewcommand{\vec}[1]{\mathbf{#1}}

\newcommand{\centercell}[1]{\multicolumn{1}{c}{#1}}
\newcolumntype{x}[1]{>{\centering\arraybackslash\hspace{0pt}}m{#1}}
\newcolumntype{y}[1]{>{\centering\arraybackslash\hspace{0pt}}m{#1\textwidth}}
\newcolumntype{M}{>{\centering\arraybackslash}m{1cm}}

\DeclareMathOperator*{\argmin}{arg\,min}
\DeclareMathOperator*{\argmax}{arg\,max}

\newcommand{\BigO}[1]{\ensuremath{\operatorname{O}\left(#1\right)}}
\newcommand{\centroid}{\mathscr{C}}
\newcommand{\iring}{\mathscr{R}}

% aligned
\renewcommand{\d}{\delta}
\newcommand{\vempty}{\mathscr{E}}
\newcommand{\e}{\epsilon}
\newcommand{\dist}{\mathtext{dist}}
\newcommand{\ball}{\mathscr{B}}


\newcommand{\hgt}{22mm}

%%--------------------------------------------------
%% Theorem environments (amsthm package required)
%%--------------------------------------------------
%% \theoremstyle{plain} %% This is the default
%\newtheorem{theorem}{Theorem}
%\newtheorem{lemma}{Lemma}
%\newtheorem{corollary}{Corollary}
%\newtheorem{conjecture}{Conjecture}
%\newtheorem{crit}{Criterion}
%\newtheorem{condition}{Condition}
%\newtheorem{fact}{Fact}

%\theoremstyle{definition}
%\newtheorem{defn}{Definition}[section]

%\theoremstyle{remark}
%\newtheorem{rem}{Remark}[section]
%\newtheorem*{notation}{Notation}

%\numberwithin{equation}{section}

%--------------------------------------------------
% Tight enumerations
%--------------------------------------------------
\newenvironment{tightenumerate}{
\begin{enumerate}
  \setlength{\itemsep}{1pt}
  \setlength{\parskip}{0pt}
  \setlength{\parsep}{0pt}
}{\end{enumerate}
}
\newenvironment{tightitemize}{
\begin{itemize}
  \setlength{\itemsep}{1pt}
  \setlength{\parskip}{0pt}
  \setlength{\parsep}{0pt}
}{\end{itemize}
}

%--------------------------------------------------
% Add figs to graphics path
%--------------------------------------------------
\graphicspath{{./}{./figs/}}

\newcolumntype{H}{@{}>{\lrbox0}l<{\endlrbox}}

%\newcommand{\container}[2]{container(#1, #2)}
\newcommand{\container}[1]{container(#1)}

\newcommand{\directAncestors}[1]{direct\_ancestors(#1)}

\newcommand{\lcp}{\textit{lcp}}

\newcommand{\shortcite}[1]{\cite{#1}}

\renewcommand{\paragraph}[1]{\noindent \textbf{#1}}

